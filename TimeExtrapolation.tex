In the successor generator step a time extrapolation is used. This extrapolation step reduces the number of DBMs created and makes sure that this number is finite. The most coarse abstraction as described in ~\cite{Behrmann2004} is used. This extrapolation reduces the number of zones that are explored significantly. It also makes that less improvements can be made on the representation of the zones, for some models all states are extrapolated to the same zone, so nothing interesting happens at the timed side of the model any more. In opaal this algorithm is implemented in such a way that all \uppaal{} locations are always read. The maximum extrapolation is based on the values of these locations. Only if there is no difference between all values for a certain location, it is not needed to read this. This results into an densely populated dependency matrix. 