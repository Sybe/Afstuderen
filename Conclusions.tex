The first goal of this project was to build a symbolic model-checker for timed automata in \ltsmin{}. This has succeeded, we have a model-checker which uses the opaal language front-end for \uppaal{} models, and the symbolic back-end of \ltsmin{}, using either the LDD or the new DDD package. This has all been achieved without changing the \pins{} structure. We only added one call to it which returns the number of discrete variables a model has. 

The experiment results were not what we hoped for. The results are slower than both \uppaal{}, and the explicit-state tool that was already implemented in \ltsmin{}. We were not able to replicate the results that were achieved earlier~\cite{ddds}. This can be explained by either the different structure of our model-checker or by the improvements that have been made by \uppaal{} since then~\cite{bbdlpw-ftrtft02}. 

One of the most fundamental problems we see are the densely filled dependency matrices. This makes it much harder to find good reorderings for symbolic structures. From our perspective this is also one of the key factors why partial order reduction for timed automata is a real challenge. Only when sparser dependency matrices can be achieved, the partial order reduction in \ltsmin{} can be used effectively. 

We have proposed a number of improvements that can be made to the DDD structure. Or even a complete overhaul of the DDDs by changing the semantics of the diagram. All of these improvements can be built upon the structure we created. With these improvements we hope that a symbolic model-checker can be built that can really compete with \uppaal{} and other model-checkers for timed automata. 

We sticked as much as possible to the LDD design of Sylvan. This to use all of the optimizations that have already been created. On some points we expect better results when we step away from this design. Especially the skipping of levels in a diagram seems to be a serious issue, as this can reduce the size of the diagram significantly. Doing this will require some extra effort, as important parts of Sylvan, as the hashtable storing all nodes, cannot be used directly. 