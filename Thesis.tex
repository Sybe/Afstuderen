\documentclass[11pt]{article}
\usepackage{cite}
\usepackage{amsthm}
\usepackage{amsfonts}

\begin{document}

\newtheorem{mydef}{Definition}
\title{My Article}
\author{Sybe van Hijum}
\date{Today}
\maketitle

\section{Introduction}
Timed Automata~\cite{Alur1994183} are a widely used modelling formalism. A recent usage of this formalism is the modelling of biological signalling pathways~\cite{DBLP:conf/bibe/SchivoSWCVKLPP12}. This leads however to large state spaces, and sometimes to models that are too large to handle by conventional methods. Therefore the ANIMO~\cite{DBLP:conf/bibe/SchivoSWCVKLPP12} tool at this time uses simulation of models and not complete state space generation and property checking.
\\BDDs(Binary Decision Diagrams)~\cite{?} and variations like LDDs(List Decision Diagrams)~\cite{so62465} and MDDs(Multi-valued Decision Diagrams)~\cite{129849} have proven their value in model checking algorithms. Due to advances in this field models with much larger state spaces can be explored on the same machine. This progress has not translated directly to more efficient methods for Timed Automata. Several methods have been proposed, like CDDs(Clock Difference Diagrams)~\cite{BRICS19491}, CMDs(Constraint Matrix Diagrams~\cite{5702245}, CRDs(Clock Restriction Diagrams)~\cite{crds} and DDDs(Difference Decision Diagrams)~\cite{Møller200188}. All of these methods show some extra difficulties or time complexity over BDDs or some limitations.
\\LTSmin~\cite{eemcs18152,ltsmin-mc:nmf2011} is a language independent on the fly model checker with several algorithmic backends. Its symbolic backend uses LDDs to both represent the state space and the transition relations of models. LTSmin has a language module for the UPPAAL~\cite{UPPAAL} through the opaal~\cite{opaal} lattice model checker. For this language at this time, only the multicore backend can be used~\cite{eemcs21972}. This multicore approach showed efficient enough to compete with the latest version of the UPPAAL model checker. It showed significant speedups on multicore machines, at the cost of some memory increase however.
\\The symbolic backend of LTSmin provides both a memory reduction by using LDDs and a speedup by using multi-threaded search algorithms and the multi-threaded LDD package Sylvan~\cite{sylvan}. Using this together with the UPPAAL language frontend will hopefully result in a modelchecker that can compete both on time and memory consumption with the UPPAAL model checker. We propose a method that uses LDDs to represent both the discrete states as the clocks and that uses DBMs~\cite{dbmorig, bengtsson2002clocks} only in the next state generation. This way we can combine existing techniques to build a complete modelchecker. It will also remain possible to use the other techniques that LTSmin has, such as transition caching, variable reordering and LTL checking. This will result in a complete modelchecker for Timed Automata.

\section{Related Work}

\begin{mydef}[Timed Automata~\cite{eemcs21972}]
\label{def:TA}
An extended timed automaton is a 7-tuple A = $<L, C, Act, s_0, \rightarrow, I_c>$ where
{\renewcommand\labelitemi{--}
	\begin{itemize}
		\item L is a finite set of locations, typically denoted by $l$
		\item C is ia finite set of clocks, typically denoted by c
		\item Act is a finite set of actions
		\item $s_0 \in$ L is the initial location
		\item $\rightarrow \subseteq L \times G(C) \times Act \times 2^C \times L$ is the (non-deterministic) transition relation. We noramlly write l $\stackrel{g,a,r}{\longrightarrow}$ l' for a transition., where l is the source location, g is the guard over the clocks, a is the action, and r is the set of clocks reset.
		\item $I_C : L \rightarrow G(C)$ is a function mapping locations to downwards closed clock invariants.
	\end{itemize}
}
\end{mydef}

\begin{mydef}[Network of timed automata~\cite{eemcs21972}]
\label{def:networkTA}
Let Act = $\{ch!,ch?|ch \in Chan\} \cup \{\tau\}$ be a finite set of actions, and let C be a finite set of clocks. Then the parallel composition of extended timed automata $A_i = (L_i, C, Act, S^i_0, \rightarrow_{i}, I^i_C)$ for all $1 \leq i \leq n$, where $n \in \mathbb{N}$, is a network of timed automata, denoted $A = A_1||A_2||..||A_n$.
\end{mydef}

\subsection{Zones}
For basic transition systems the state space can grow exponentially for the size of the system. For Timed Automata the growth can become even larger and in some cases become unbounded. To tackle this problem most model checkers use a notion of zones for the representation of time. A zone can be seen as a set of constraints over the clocks C of the form $c_i \sim x$ and $c_i - c_j \sim x$ where $\sim  \in \{<, \leq, =, \geq, >\}$ and $x \in \mathbb{N}$. To represent these zones several data structures have been developed. One of the most common used structures are Difference Bound Matrices(DMB's)~\cite{bengtsson2002clocks}.\\ 
These matrices use both a column and a row for each clock, and on each position $(i,j)$ an upperbound on the difference between the clocks $c_i$ and $c_j$ is given in the form $c_i - c_j \preceq x$ where $\preceq \in \{<, \leq\}$ and $x \in \mathbb{N}$. For the constraints over the single clocks an extra clock $\mathbf{O}$ with a constant value 0 is added.

\bibliography{references}{}
\bibliographystyle{plain}
\end{document}