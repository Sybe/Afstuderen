\documentclass[11pt]{article}
\usepackage{cite}
\usepackage{amsthm}
\usepackage{amsfonts}


\begin{document}

\newtheorem{mydef}{Definition}
\title{My Article}
\author{Sybe van Hijum}
\date{Today}
\maketitle

Blablabla said Nobody ~\cite{eemcs21972}. Maar kees zei ~\cite{eemcs18152}



\begin{mydef}[Timed Automata]
\label{def:TA}
An extended timed automaton is a 7-tuple A = $<L, C, Act, s_0, \rightarrow, I_c>$ where
{\renewcommand\labelitemi{--}
	\begin{itemize}
		\item L is a finite set of locations, typically denoted by $l$
		\item C is ia finite set of clocks, typically denoted by c
		\item Act is a finite set of actions
		\item $s_0 \in$ L is the initial location
		\item $\rightarrow \subseteq L \times G(C) \times Act \times 2^C \times L$ is the (non-deterministic) transition relation. We noramlly write l $\stackrel{g,a,r}{\longrightarrow}$ l' for a transition., where l is the source location, g is the guard over the clocks, a is the action, and r is the set of clocks reset.
		\item $I_C : L \rightarrow G(C)$ is a function mapping locations to downwards closed clock invariants.
	\end{itemize}
}
\end{mydef}

\begin{mydef}[Network of timed automata]
\label{def:networkTA}
Let Act = $\{ch!,ch?|ch \in Chan\} \cup \{\tau\}$ be a finite set of actions, and let C be a finite set of clocks. Then the parallel composition of extended timed automata $A_i = (L_i, C, Act, S^i_0, \rightarrow_{i}, I^i_C)$ for all $1 \leq i \leq n$, where $n \in \mathbb{N}$, is a network of timed automata, denoted $A = A_1||A_2||..||A_n$.
\end{mydef}

\bibliography{references}{}
\bibliographystyle{plain}
\end{document}