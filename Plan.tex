We will implement first a method that will use the best of both worlds, the efficient algorithms from de DBMs and the memory efficiency of a symbolic approach. We will use the DBMs in the state exploration such that we can find a canonical representation of the clock zone of a newly explored state quite easily. For the symbolic representation of the state space, including the clock zones, and the transition relations, we will use normal BDDs. The DBMs will be flattened and put directly into the state vector and can then be handled by the symbolic BDD backend. Therefore both the efficient algorithms and the memory efficient representation can be used. A downside to this approach is that a zone subsumption check is not possible anymore, as only equalities and no inequalities can be checked on BDDs, resulting in revisiting of some states. Further we will focus on efficient orderings of the BDDs, as both clock zones and states are contained in a single structure. We will also use this new method with the existing multicore tool, such that we can still use the subsumption check that is implemented in LTSmin. Afther that we will continue towards a DDD model checker. First we will use the DDDs as the state space representation and still use the language module using the DBMs. We have not been able to find any literature on the combination of these techniques. There might be a significant memory improvement possible here. Eventually we aim at a complete symbolic solution with more operations on the DDD, such as the progress of time, then we can have a language module which does not use the DBMs any more. We will compare the different approaches we implement extensively to each other. All of these approaches will be implemented in the LTSmin toolset. This way we can really compare the methods and not just the tools.

Alongside this we will also have to make the opaal PINS work with the UPPAAL models generated by ANIMO. The current versions doe not work together because of global variables are used in the system declaration in the generated model, and this is a feature that opaal does not support. We can make this work by either changing the models generated by ANIMO or by extending the opaal PINS. At this time we do not know the best solution for this problem.


\subsection{Questions}
For the research we will state a couple of research questions:
{\renewcommand\labelitemi{--}
	\begin{itemize}
		\item Is the combination of BDDs and flattened DBMs an efficient method for symbolic reachability analysis of timed automata? Both on memory usage and speed.
		\item Can improvements be achieved by using different orderings? Both by changing the order of only the clock variables and by mixing the clock and state variables.
		\item Is the new language module needed for the symbolic approach also usable for the multicore approach with subsumption?
		\item Can the BDD approach be generalized towards a method using DDDs?
		\item Is a fully symbolic reachability analysis using DDDs more efficient than the combination of DDDs and DBMs, both on memory and speed?
	\end{itemize}
}

\subsection{Algorithms}
To create a DDD library we will implement a number of functions over DDDs. We will limit the functions to the ones needed for this purpose. Therefore it will not become a complete DDD package. One of the core operations on DDDs is the apply operation. This operation takes two DDDs and a binary operator and combines the two DDDs according to the operator. The apply function for DDDs is a generalisation of the function for BDDs. In ~\cite{ddds} a general definition of the algorithm is given. We turned this more mathematical definition into an algorithm. In algorithm \ref{alg:union} we give the pseudo-code for the apply function with the or operator, which we renamed to the Union function. All functions rely on a Mk function which checks if the node needed already exists, and otherwise creates a new node.
The subsumption check, which we lost in the BDD approach, will be possible again with DDDs. This will be the same check as a state membership in an LDD. The only difference is that no equality, but upper bounds will be checked. Pseudo-code for this algorithm is given in algorithm \ref{alg:contain}. If we combine DDDs with LDDs, only the correct check has to be adapted, the algorithm will remain the same.

\begin{comment}
In BDDs it is simple to test for containment of a state. This can be done in linear time over the number of variables. For zones in DDDs this can be a harder question, as a zone can be contained in some larger zone in the DDD. For this we need an extension to the normal LDD algorithm, that can handle these zones. The main difference is that on some nodes both the high and the low edge can lead to a satisfying path. Algorithm \ref{alg:containOLD} tests if a zone $z$ is contained in the DDD rooted at $v$.

\begin{algorithm}
\caption{Zone containment for DDDs}\label{alg:containOLD}
\begin{algorithmic}[1]
\Procedure{Contains}{$v, z$}
	\If{$v \in {0,1}$} 
		\Return{$v$} 
	\EndIf
	\If{z[$var(v)$] correct in $v$}
		\If{\Call{Contains}{$high(v), z$}} 
			\Return{\True}
		\Else{ 
			\Return{\Call{Contains}{$low(v), z$}}} 
		\EndIf
	\Else{ 
		\Return{\Call{Contains}{$low(v), z$}}} 
	\EndIf
\EndProcedure
\end{algorithmic}
\end{algorithm}

Where the correct in check will evaluate if the upper-bound in the zone is lower than the upper bound in the node $v$. As we will use maximal sharing of subtrees, some nodes might be revisited. Once we revisit a node, we know that none of it subtrees will lead to a correct evaluation. Therefore we could use a visited flag, to save computation time. The worst case running time of the algorithm will be $\mathcal{O}(|v|)$, where $|v|$ is the number of nodes in the DDD. This can increase the running time of the state space generation significantly, as this algorithm will need to be ran every time a new state is found. It will however save memory, as no zones will be added that are already in the DDD. As the main goal of a symbolic approach is the reduction of memory, we think this is the better solution, over a less time consuming approach which would lead to more memory consumption. The algorithm will not work correctly when the discrete variables are also translated into DDD nodes, as the value in those cases needs to be correct, and an higher upper bound will not work. For the mix of DDD and LDD nodes we will need to change the algorithm to algorithm \ref{alg:containLDD}. This algorithm checks if a node is of type DDD. If it is, the old algorithm will be executed normally, if not, then it is an LDD node and in case of a correct evaluation only the high edge will be explored. The low edge will in that case never lead to a correct evaluation of the state.

\begin{algorithm}
\begin{algorithmic}[1]
\caption{Zone containment for mixed diagram}\label{alg:containLDD}
\Procedure{Contains}{$v, z$}
	\If{$v \in {0,1}$} 
		\Return{$v$} 
	\EndIf
	\If{z[$var(v)$] correct in $v$}
		\If{$type(v)$ is DDD} 
			\If{\Call{Contains}{$high(v), z$}} 
				\Return{\True}
			\Else{ 
				\Return{\Call{Contains}{$low(v), z$}}} 
			\EndIf
		\Else{
			\Return{\Call{Contains}{$high(v), z$}}}
		\EndIf 
	\Else{ 
		\Return{\Call{Contains}{$low(v), z$}}} 
	\EndIf
\EndProcedure
\end{algorithmic}
\end{algorithm}

\end{comment}

\begin{algorithm}
\begin{algorithmic}[1]
\caption{Union}\label{alg:union}
\Procedure{Union}{$v1, v2$}
	\If{$v1 = v2$} 
		\Return{$v1$} 
	\ElsIf{$v1 =$ \False}
		\Return{$v2$}
	\ElsIf{$v2 =$ \False}
		\Return{$v1$}
	\ElsIf{$var(v1) \prec var(v2)$}
		\State $high \gets$ \Call{Union}{$high(v1), v2$}
		\State $low \gets$ \Call{Union}{$low(v1), v2$}
		\State $result \gets$ \Call{Mk}{$cstr(v1), high, low$} 
	\ElsIf{$var(v2) \prec var(v1)$}
		\State $high \gets$ \Call{Union}{$high(v2), v1$}
		\State $low \gets$ \Call{Union}{$low(v2), v1$}
		\State $result \gets$ \Call{Mk}{$cstr(v2), high, low$} 
	\ElsIf{$v1 \prec v2$}
		\State $high \gets$ \Call{Union}{$high(v1), high(v2)$}
		\State $low \gets$ \Call{Union}{$low(v1), v2$}
		\State $result \gets$ \Call{Mk}{$cstr(v1), high, low$}
	\ElsIf{$v2 \prec v1$}
		\State $high \gets$ \Call{Union}{$high(v1), high(v2)$}
		\State $low \gets$ \Call{Union}{$v1, low(v2)$}
		\State $result \gets$ \Call{Mk}{$cstr(v2), high, low$}
	\ElsIf{$v1 = v2$}
		\State $high(v1) \gets$ \Call{Union}{$high(v1), high(v2)$}
		\State $low(v1) \gets$ \Call{Union}{$low(v1), low(v2)$}
		\State $result \gets$ \Call{Mk}{$cstr(v1), high, low$}
	\EndIf
	\State \Return $result$
\EndProcedure
\end{algorithmic}
\end{algorithm}

\begin{algorithm}
\caption{Zone containment for DDDs}\label{alg:contain}
\begin{algorithmic}[1]
\Procedure{Contains}{$v, z$}
	\If{$v \in {0,1}$} 
		\Return{$v$} 
	\ElsIf{z[$var(v)$] correct in $v$}
		\State\Return{\Call{Contains}{$high(v), z$}} 
	\Else{ 
		\Return{\Call{Contains}{$low(v), z$}}} 
	\EndIf
\EndProcedure
\end{algorithmic}
\end{algorithm}




\subsection{To do}
In this section we describe all things that need to be implemented to make model checking with a certain diagram possible. 

To make symbolic model checking work we need to change the opaal PINS. The PINS currently uses a pointer to a DBM. For the new approach we will put the values of the DBM directly into the state vector. This will increase the size of the state vector. All other references to the types and values of the state vector entries will need to be changed also. 

To make symbolic variable reordering possible we will need to partition the next state function and create a dependency matrix. In the code the next state function is already split up per transition, but in a single transition group. Splitting this into multiple transition groups should not be too hard. In this same step also the dependency matrices need to be created and made as sparse as possible.

To combine the new PINS with the multicore LTSmin backend the subsumption check will need to be changed. This check now relies on a pointer to a DBM, but it will now get the complete DBM, or state vector. Here the search algorithm or the subsumption check will need to know which variables are zone variables. It will also occur that different states will have the same discrete variables, but different zone variables, these all need to be checked for the subsumption check.

For the combination with the multicore backend also the data structure will need to be adapted. The current structure stores a discrete state together with a set of pointers to DBMs. In the new situation each pair of discrete state and DBM will be stored explicitly.

For the DDD the diagram will need to be able to identify the zone variables from the discrete state variables in the state vector. This will need an extra function in the PINS interface recognizing the different types of variables.  

For the DDD representation the values from the DBM will need to be translated to useful variables as ordering of the DDD is based on the values. Also to check for inequalities and set containment the meaningful values are needed. In the current DBM library both the value and the operator are saved in a single 32 bit integer. The DDD will need to know the value and the operator separately.

To make an efficient DDD representation, we need to mix these DDD nodes with BDDs or LDDs for the discrete variables. For this diagram this has not been done before, we can only look at the CRD solution. Algorithms probably need some adaptation for these multiple types of nodes. Also the possibility and results of the mixing of discrete and zone variables need to be researched. It might be that it functions better when the two types of variables remain separated, or when they are mixed.

\subsection{Planning}
In the table below we have put all actions that need to be done into tasks. In the second column we put which tool should work correctly for the opaal language module after the task. This will give us intermediate points on which we can test the work that has been done to that point.

%\begin{table}[]
%\centering
%\caption{Tasks}
\label{table:tasks}
\label{my-label}
\begin{tabular}{@{}ll@{}}
\toprule
Task                                            & Needs to function                                                           \\ \midrule
Flatten the DBMs                                & Symbolic tool                                                               \\
Partition the next state function               & Symbolic tool                                                               \\
Create dependency matrices                      & \begin{tabular}[c]{@{}l@{}}Symbolic tool\\ variable reordering\end{tabular} \\
Change subsumption check for multicore approach &                                                                             \\
Other adaptions multicore approach              & Multicore tool                                                              \\
Create minimal DDD library                      &                                                                             \\
Combine DDD with LDD                            &                                                                             \\
Extend language module for DDD approach         & Symbolic tool                                                               \\
Benchmark tests for multiple approaches         &                                                                             \\
Test with ANIMO models                          &                                                                             \\
Extend DDD library for fully symbolic approach  &                                                                             \\
Change language module for fully symbolic DDDs  & Symbolic tool                                                               \\
Benchmark Testing fully symbolic approach       &                                                                             \\
Writing Report                                  &                                                                             \\ \bottomrule
\end{tabular}
%\end{table}