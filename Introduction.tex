Timed automata~\cite{Alur1994183} is a widely used modelling formalism. A recent usage of this formalism is the modelling of biological signalling pathways~\cite{DBLP:conf/bibe/SchivoSWCVKLPP12}. ANIMO is a tool that generates these timed automata from biological signalling pathways models. This leads however to large state spaces, and sometimes to models that are too large to handle by conventional methods. Therefore better model checking techniques for timed automata, that can handle larger state spaces are needed. We look into symbolic algorithms for timed automata.

BDDs (Binary Decision Diagrams)~\cite{Akers:1978:BDD:1310167.1310815,1676819} and variations like LDDs (List Decision Diagrams)~\cite{so62465} and MDDs (Multi-valued Decision Diagrams)~\cite{129849} have proven their value in model checking algorithms. Due to advances in this field, models with much larger state spaces can be explored on the same machine. This progress has not been translated directly to more efficient methods for timed automata. Several methods have been proposed, like CDDs (Clock Difference Diagrams)~\cite{BRICS19491}, CMDs (Constraint Matrix Diagrams)~\cite{5702245}, CRDs (Clock Restriction Diagrams)~\cite{crds} and DDDs (Difference Decision Diagrams)~\cite{ddds, ddd-datastructure-99}. All of these methods show some extra difficulties or limitations over BDDs. Also after their introduction they have not been developed further.

LTSmin~\cite{eemcs18152,ltsmin-mc:nmf2011} is a language independent on the fly model checker with several algorithmic back-ends. Its symbolic back-end uses BDDs to both represent the state space and the transition relations of models. These BDDs are generated on the fly by the search algorithms. LTSmin has a language module for the UPPAAL~\cite{UPPAAL} through the Opaal~\cite{opaal} lattice model checker. Through this module UPPAAL models can be loaded into LTSmin. For this language currently, only the multi-core back-end can be used~\cite{eemcs21972}. This multi-core approach showed efficient enough to compete with the latest version of the UPPAAL model checker. It showed significant speedups on multi-core machines, at the cost of some memory increase however. To tackle the memory increase a combination of the Opaal front-end and the symbolic back-end could be a solution.

The symbolic back-end of LTSmin provides both a memory reduction by using BDDs and a speedup by using multi-threaded search algorithms and the multi-threaded BDD package Sylvan~\cite{sylvan}. Using this together with the UPPAAL language front-end will hopefully result in a model checker that can compete both on time and memory consumption with the UPPAAL model checker.

We will propose a symbolic reachability for timed automata that is capable of handling the models that are generated by the ANIMO tool.